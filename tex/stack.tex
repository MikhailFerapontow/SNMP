Выбор языка программирования Python для разработки системы сбора и хранения метрик с использованием протокола SNMP обоснован рядом преимуществ данного языка. Python - это высокоуровневый язык программирования, который отличается простотой в изучении, читаемостью кода и мощными инструментами для разработки. Его синтаксис интуитивно понятен, что делает процесс написания кода более эффективным и быстрым. Кроме того, Python имеет обширную стандартную библиотеку, которая включает в себя множество полезных модулей для работы с сетями, обработки данных и других задач, что упрощает разработку и сокращает время на написание кода с нуля.

Для работы с протоколом SNMP в Python была выбрана библиотека PySNMP\cite{pysnmpSNMPLibrary}. Её преимущества заключаются в том, что она предоставляет простой и удобный интерфейс для работы с SNMP протоколом, обеспечивая полный функционал для отправки запросов к агентам SNMP, получения данных, а также для управления устройствами. PySNMP предоставляет высокоуровневые абстракции для работы с SNMP, что позволяет сосредоточиться на разработке бизнес-логики приложения, минимизируя сложности взаимодействия с протоколом. Кроме того, PySNMP является популярной и активно поддерживаемой библиотекой в сообществе Python, что обеспечивает доступность документации, обновлений и поддержку со стороны разработчиков. Все эти факторы делают PySNMP превосходным выбором для реализации функционала сбора и хранения метрик с использованием протокола SNMP в рамках разрабатываемой системы.