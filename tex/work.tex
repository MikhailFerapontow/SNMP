\subsection*{Конфигурация SNMP}

Для работы с SNMP нужно установить клиента и демона. Для arch linux существует пакет
net-snmpd, который включает в себя всё необходимое.

SNMP v3 добавляет безопасность и шифрованную аутентификацию/коммуникацию.
Он использует схему конфигурации в файле /etc/snmp/snmpd.conf и дополнительную
конфигурацию в файле /var/net-snmp/snmpd.conf.

Для первого конфигурационного файла в /etc/ можно записать непосредственно
в него или использовать мастер настройки snmpconf:

\begin{verbatim}
$ mkdir /etc/snmp/
$ echo rouser read_only_user >> /etc/snmp/snmpd.conf
$ snmpconf -g basic_setup
\end{verbatim}

Мы воспользуемся мастером настройки snmpconf.

\subsubsection*{Демон}

Запускаем службу snmpd.service.
\begin{verbatim}
$ sudo systemctl start snmpd
\end{verbatim}

\subsubsection*{Тестирование}
Если используется SNMP 1 или 2c, используйте одну из следующих команд для тестирования конфигурации:

\begin{verbatim}
$ snmpwalk -v 2c -c read_only_community_string localhost | less
\end{verbatim}

Мы смогли получить значения OID системы, значит конфигурация прошла успешно.

\subsection*{Определение OID для метрик}

Прежде чем приступить к разработке приложения, необходимо определить OID для метрик.
Мы будем отслеживать следующие метрики:
\begin{itemize}
    \item Нагрузку процессора.
    \item Температуру процессора.
    \item Использование RAM.
    \item Исходящий и входящий интернет трафик.
\end{itemize}

\subsubsection*{Нагрузка процессора}

\begin{enumerate}
    \item \textbf{.1.3.6.1.4.1.2021.10.1.3.1} -- нагрузка процессора за 1 минуту
    \item \textbf{.1.3.6.1.4.1.2021.10.1.3.2} -- нагрузка процессора за 5 минут
    \item \textbf{.1.3.6.1.4.1.2021.10.1.3.3} -- нагрузка процессора за 15 минут
\end{enumerate}

Данная метрика показывает общую нагрузку процессора в зависимости от количества ядер, где
100\% нагрузка равна количествую ядер. Например, в одноядерной системе нагрузка 1 будет соответсвовать
100\% нагрузке, когда же в двухядерной системе нагрузка будет равна 50\%.

\subsubsection*{Температура процессора}

К сожалению, нельзя узнать температуру процессора через SNMP "из коробки". Поэтому
мы воспользуемся системой мониторинга \textbf{lm\_sensors} \cite{archlinuxLm}. Она создаст своё MIB хранилище
и мы получим доступ к нему через SNMP.

\begin{enumerate}
    \item \textbf{LM-SENSORS-MIB::lmTempSensorsValue.1} -- температура процессора
\end{enumerate}

Как можно заметить OID может представляться как числовым значением, так и текстовым.

\subsubsection*{Интернет трафик}

\begin{enumerate}
    \item \textbf{IF-MIB::ifInOctets} -- входящий интернет трафик
    \item \textbf{IF-MIB::ifOutOctets} -- исходящий интернет трафик
\end{enumerate}

Данная метрика показывает какое количество октетов (байт) передалось из или в интерфейс с момента
начала работы системы. Так как в системе может быть большое количество интерфейсов, то мы провёдем
операцию под названием \textbf{snmpwalk} и соберём значения со всех интерфейсов.

\subsubsection*{Использование RAM}

\begin{enumerate}
    \item \textbf{UCD-SNMP-MIB::memTotalReal.0} -- общее количество RAM на устройстве
    \item \textbf{UCD-SNMP-MIB::memAvailReal.0} -- доступное количество RAM для использования
    \item \textbf{UCD-SNMP-MIB::memCached.0} -- закеширование количество RAM на устройстве
\end{enumerate}

\subsection*{Разработка приложения}

Когда мы узнали все нужные для работы OID, мы можем приступить к разработке приложения.
Библиотека \textbf{PySNMP} предоставляет простой и удобный интерфейс для работы с SNMP протоколом.
Был написан класс SNMPClient.

\lstinputlisting[language=python]{./snmp/snmp/snmp.py}