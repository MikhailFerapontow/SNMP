Simple Network Management Protocol (SNMP) представляет собой стандартный протокол управления сетью\cite{Mauro2005},
который используется для мониторинга и управления сетевыми устройствами в информационных технологиях.
Он разработан для обеспечения стандартизированного и эффективного способа сбора информации о состоянии
устройств в сети, а также для удаленного управления этими устройствами.


При использовании SNMP один или несколько административных компьютеров, именуемых менеджерами, осуществляют мониторинг или управление группой хостов или устройств в компьютерной сети. На каждом управляемом устройстве установлена постоянно работающая программа, известная как агент, которая через SNMP передает информацию менеджеру.

Сети, управляемые протоколом SNMP, состоят из трех основных компонентов\cite{snmp}:

\begin{enumerate}
    \item Управляемое устройство;
    \item Агент - программное обеспечение, работающее на управляемом устройстве или на устройстве, подключенном к интерфейсу управления управляемого устройства;
    \item Система сетевого управления (Network Management System, NMS) - программное обеспечение, взаимодействующее с менеджерами для поддержки комплексной структуры данных, отражающей состояние сети.
\end{enumerate}

Управляемое устройство представляет собой элемент сети (оборудование или программное обеспечение), которое реализует интерфейс управления (не обязательно SNMP), позволяющий однонаправленный (только для чтения) или двунаправленный доступ к конкретной информации об элементе. Управляемые устройства обмениваются этой информацией с менеджером. Они могут включать в себя разнообразные устройства, такие как маршрутизаторы, серверы доступа, коммутаторы, мосты, концентраторы, IP-телефоны, IP-видеокамеры, компьютеры-хосты, принтеры и т.д.

Агент - это программный модуль сетевого управления, работающий на управляемом устройстве или на устройстве, подключенном к интерфейсу управления управляемого устройства. Он обладает локальным знанием управляющей информации и преобразует эту информацию в формат, совместимый с SNMP, или извлекает ее из этого формата (медиация данных).

Система сетевого управления (NMS) включает приложение, которое отслеживает и контролирует управляемые устройства. NMS выполняют основную часть обработки данных, необходимых для сетевого управления. В управляемой сети может присутствовать одна или несколько NMS.

\begin{figure}[H]
    \centering
    \includegraphics*[width=0.8\textwidth]{img/SNMP_communication_principles_diagram.PNG}
    \caption{Принципы коммуникации SNMP}
\end{figure}

Основными компонентами SNMP являются управляющая станция (SNMP менеджер), агенты SNMP и информационная база данных
управляемых объектов (MIB - Management Information Base). MIB представляет собой структурированный набор данных,
определяющих атрибуты и параметры управляемых объектов, которые могут быть мониторены и управляемы через SNMP.
Каждое устройство, поддерживающее SNMP, имеет свою собственную MIB, определяющую доступные для мониторинга и управления параметры.

Преимущества SNMP включают в себя его стандартизированный характер, простоту использования и эффективность в мониторинге
и управлении сетевыми устройствами. Благодаря SNMP администраторы сети могут получать информацию о производительности
сети, загрузке устройств, использовании ресурсов, количестве ошибок в передаче данных и других параметрах, что
помогает в выявлении и устранении проблем в сети.

В целом, SNMP является важным инструментом для администрирования сетей, обеспечивая возможность мониторинга и управления
сетевыми устройствами, что способствует эффективной работе сети и обеспечивает оперативное реагирование на проблемы.