SNMP (Simple Network Management Protocol) \cite{Mauro2005} представляет собой протокол на прикладном уровне, разработанный Советом по архитектуре Интернета (IETF) для обмена данными между сетевыми устройствами в сетях любого размера. Он позволяет системным администраторам мониторить, контролировать производительность сети и изменять конфигурацию устройств без необходимости ввода команд вручную.

Архитектура SNMP включает в себя несколько компонентов:

\begin{enumerate}
    \item Сетевая станция управления (NMS): Она представляет собой программное обеспечение, которое мониторит и управляет устройствами в сети. NMS взаимодействует с агентами устройств через протокол SNMP.
    \item Агенты: Это программное обеспечение, установленное на управляемых устройствах. Оно собирает информацию о состоянии устройства и передает её NMS.
    \item Мастер-агенты: Программы, которые связывают сетевые менеджеры и субагенты, а также анализируют запросы от NMS и управляют передачей данных между ними.
    \item Субагенты: Это программное обеспечение, поставляемое вендором вместе с управляемыми устройствами. Они собирают информацию об устройстве и передают её мастер-агенту.
    \item Управляемые компоненты: Это сетевые устройства или программное обеспечение с установленными агентами. Они могут быть различными, от маршрутизаторов и коммутаторов до IP-видеокамер и антивирусных программ.
\end{enumerate}

\begin{figure}[H]
    \centering
    \includegraphics*[width=0.6\textwidth]{./img/snmp-system.png}
\end{figure}

Для организации обмена данными между NMS и агентами используются специальные единицы данных, называемые Protocol Data Unit (PDU)\cite{Alani2014}. Существует семь типов PDU, таких как:

\begin{enumerate}
    \item \textbf{GET} -- запрос данных
    \item \textbf{SET} -- изменение данных
    \item \textbf{RESPONSE} -- ответ на запрос
    \item \textbf{TRAP} -- уведомление об событии
    \item \textbf{GETBULK} -- запрос агенту на извлечение с устройства массива данных. Это улучшенный вариант запроса GETNEXT.
    \item \textbf{INFORM} -- сообщение, аналогичное TRAP, но с подтверждением получения. Агент будет отправлять уведомление, пока менеджер не подтвердит, что оно дошло.
\end{enumerate}

Изначально SNMP разрабатывался для управления интернетом, однако его гибкая архитектура позволяет мониторить и управлять
всеми сетевыми устройствами с помощью единого интерфейса. Наиболее распространенной версией является SNMPv2c.